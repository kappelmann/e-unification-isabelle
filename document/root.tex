\documentclass[11pt,a4paper]{article}
\usepackage[T1]{fontenc}
\usepackage{isabelle,isabellesym}

% this should be the last package used
\usepackage{pdfsetup}

% urls in roman style, theory text in math-similar italics
\urlstyle{rm}
\isabellestyle{it}

\begin{document}

\title{Unification Utilities for Isabelle/ML}
\author{Kevin Kappelmann}
\maketitle

\begin{abstract}
This article provides various unification utilities for Isabelle/ML, most prominently:
\begin{enumerate}
\item First-order and higher-order pattern
\href{https://en.wikipedia.org/wiki/Unification_(computer_science)#E-unification}{E-unification}
and E-matching.
While unifiers in Isabelle/ML only consider the $\alpha\beta\eta$-equational theory of the $\lambda$-calculus,
unifiers in this article
may take an extra background theory, in the form of an equational prover, into account.
For example, the unification problem $n + 1 \equiv {}?m + Suc 0$
may be solved by providing a prover for the background theory $\forall n.\ n + 1 \equiv n + Suc 0$.
\item Tactics, methods, and attributes with adjustable unifiers (e.g.\ resolution, fact, assumption, OF).
\item A generalisation of unification hints~\cite{unif-hints}.
Unification hints are a flexible extension for unifiers.
Among other things, they can be used for reflective tactics,
to provide canonical unification instances,
or to simply strengthen the background theory of a unifier in a controlled manner.
\item Simplifier integration for e-unifiers.
\item Practical combinations of unification algorithms, e.g. a combination of first-order and
higher-order pattern unification.
\item A hierarchical logger for Isabelle/ML,
including per logger configurations with log levels, output channels, message filters.
\end{enumerate}
While this entry works with every object logic,
some extra setup for Isabelle/HOL and application examples are provided.
All unifiers are tested with SpecCheck~\cite{speccheck}.
\end{abstract}

\tableofcontents

% include generated text of all theories
\input{session}

\bibliographystyle{abbrv}
\bibliography{root}

\end{document}
